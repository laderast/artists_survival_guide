% Options for packages loaded elsewhere
\PassOptionsToPackage{unicode}{hyperref}
\PassOptionsToPackage{hyphens}{url}
%
\documentclass[
]{book}
\usepackage{lmodern}
\usepackage{amssymb,amsmath}
\usepackage{ifxetex,ifluatex}
\ifnum 0\ifxetex 1\fi\ifluatex 1\fi=0 % if pdftex
  \usepackage[T1]{fontenc}
  \usepackage[utf8]{inputenc}
  \usepackage{textcomp} % provide euro and other symbols
\else % if luatex or xetex
  \usepackage{unicode-math}
  \defaultfontfeatures{Scale=MatchLowercase}
  \defaultfontfeatures[\rmfamily]{Ligatures=TeX,Scale=1}
\fi
% Use upquote if available, for straight quotes in verbatim environments
\IfFileExists{upquote.sty}{\usepackage{upquote}}{}
\IfFileExists{microtype.sty}{% use microtype if available
  \usepackage[]{microtype}
  \UseMicrotypeSet[protrusion]{basicmath} % disable protrusion for tt fonts
}{}
\makeatletter
\@ifundefined{KOMAClassName}{% if non-KOMA class
  \IfFileExists{parskip.sty}{%
    \usepackage{parskip}
  }{% else
    \setlength{\parindent}{0pt}
    \setlength{\parskip}{6pt plus 2pt minus 1pt}}
}{% if KOMA class
  \KOMAoptions{parskip=half}}
\makeatother
\usepackage{xcolor}
\IfFileExists{xurl.sty}{\usepackage{xurl}}{} % add URL line breaks if available
\IfFileExists{bookmark.sty}{\usepackage{bookmark}}{\usepackage{hyperref}}
\hypersetup{
  pdftitle={An Artist's Survival Guide},
  pdfauthor={Ted Laderas},
  hidelinks,
  pdfcreator={LaTeX via pandoc}}
\urlstyle{same} % disable monospaced font for URLs
\usepackage{longtable,booktabs}
% Correct order of tables after \paragraph or \subparagraph
\usepackage{etoolbox}
\makeatletter
\patchcmd\longtable{\par}{\if@noskipsec\mbox{}\fi\par}{}{}
\makeatother
% Allow footnotes in longtable head/foot
\IfFileExists{footnotehyper.sty}{\usepackage{footnotehyper}}{\usepackage{footnote}}
\makesavenoteenv{longtable}
\usepackage{graphicx}
\makeatletter
\def\maxwidth{\ifdim\Gin@nat@width>\linewidth\linewidth\else\Gin@nat@width\fi}
\def\maxheight{\ifdim\Gin@nat@height>\textheight\textheight\else\Gin@nat@height\fi}
\makeatother
% Scale images if necessary, so that they will not overflow the page
% margins by default, and it is still possible to overwrite the defaults
% using explicit options in \includegraphics[width, height, ...]{}
\setkeys{Gin}{width=\maxwidth,height=\maxheight,keepaspectratio}
% Set default figure placement to htbp
\makeatletter
\def\fps@figure{htbp}
\makeatother
\setlength{\emergencystretch}{3em} % prevent overfull lines
\providecommand{\tightlist}{%
  \setlength{\itemsep}{0pt}\setlength{\parskip}{0pt}}
\setcounter{secnumdepth}{5}

\title{An Artist's Survival Guide}
\author{Ted Laderas}
\date{2020-12-17}

\begin{document}
\maketitle

{
\setcounter{tocdepth}{1}
\tableofcontents
}
\begin{verbatim}
## Warning: knitr output format is not HTML. Use `include_meta()` to ensure that the <meta> tags are properly included in the <head>
## output (if possible).
\end{verbatim}

\hypertarget{introduction}{%
\chapter{Introduction}\label{introduction}}

To make art in the world is difficult. Making art in an indifferent world, when you have a small audience, is even more difficult.

This book consists of essays written from 2015 to 2020 about how to continue to make art in an indifferent world. They were mostly previously published on my blog: \href{https://15people.net}{15 People}. Hopefully you will find these essays helpful as you continue on in your artistic journey.

\hypertarget{about-me}{%
\section{About Me}\label{about-me}}

I am a \href{https://15people.net}{Musician}, a \href{https://laderast.github.io}{Data scientist} and an \href{https://laderast.github.io/cv}{Educator}. I have a pretty unique view of making art.

Why should you listen to me? I have been making music and art in some form for over 28 years. I don't consider myself to be wildly successful, but I have managed to continue making art in some form or other, and I consider that mildly successful.

I have a number of releases and music and more information is available here: \url{https://15people.net/music/}.

\hypertarget{one-note}{%
\section{One Note}\label{one-note}}

Note it is called \emph{an} Artist's Survival Guide, not \emph{the} Artist's Survival Guide. As you continue in your artistic career, you will need to develop your own strategies to survive.

\hypertarget{license}{%
\section{License}\label{license}}

This book is licensed under a \href{http://creativecommons.org/licenses/by/4.0/}{Creative Commons Attribution 4.0 International license}. Please share freely, as long as you attribute me as its creator (Ted Laderas, tedladeras {[}at{]} gmail).

\hypertarget{on-sensitivity}{%
\chapter{On Sensitivity}\label{on-sensitivity}}

Sensitivity is required for art. Photographers and visual artists have a sensitivity to
light that is fundamental to their work. Rimbaud, that visionary poet, remarked
that a poet becomes a poet through a ``systematic derangement of the senses''.
Making music requires sensitivity to how slight tweaks in a track affect the overall perception of the piece. Sensitivity is also required as
empathy; our profound empathy for others is what helps make our art universal
and for everyone.

Unfortunately, this profound sensitivity to the world also makes it difficult for artists to
live in the every day.

My own sensitivity to sound and feelings make it difficult to ride the bus when
someone is loudly talking on the phone. I get drawn into conversations that I
don't want to get drawn into; soon I'm feeling their tragedies or stresses and
as a result I arrive at work an emotional wreck.

The real danger for sensitive artists in an unfeeling world is that it is often easy to
lose one's sense of self when confronted with so many overwhelming experiences.
Such experiences are draining and tire us out and keep us from making art. Thus, it is important to know
how to protect oneself from these experiences. We can't always be sensitive to
the world; otherwise we would be a bundle of raw nerves.

When they
are overwhelmed, artists and sensitive people have many strategies for escaping
these experiences. Some become emotionally withdrawn, reverting back to
childish and petulant behavior. Others seek alcohol and drugs to dull their
senses, or destructive impulses such as lashing out at others or destroying
property.

There are less antisocial and less self-destructive ways to protect yourself when you are
overwhelmed. These include the following:

\begin{itemize}
\item
  \emph{Mindfulness exercises and cognitive behaviorial therapy.} Both of these kinds of
  therapies provide you with methods to distance yourself from the problem and
  ask you why you are overwhelmed. Using these techniques can lessen the power of
  the overwhelming situation.
\item
  \emph{Isolating yourself from the problem behavior.} This can be as simple as headphones, humming a song to one's self, or going somewhere else. You must carve out such private spaces even in a public
  space like a club or performance venue.
\item
  \emph{Realizing that your sensitive side needs to be protected much like a child.} Don't
  ignore your own needs when you are feeling overwhelmed. Go somewhere else, calm
  yourself down before you deal with anything else.
\item
  \emph{Cultivating a sense of humor can help}. Not only can it help defuse these overwhelming
  situations, but it can lessen the impact. I doubt I would be alive today if it
  were not for my sense of humor.
\end{itemize}

Obviously, these are just a few solutions to dealing with being oversensitive to situations, but I find they work for me. If you have any other solutions that you find helpful, I'd like to
hear it. The important thing is that you realize that you have to nurture your
sensitive side for you to be at your best.

\hypertarget{resources}{%
\section{Resources}\label{resources}}

The two books I highly recommend are Elaine Aron's \href{https://hsperson.com/}{\emph{The Highly Sensitive Person}} and Susan Cain's \href{https://www.goodreads.com/book/show/8520610-quiet}{\emph{Quiet: The Power of Introverts in a World that Can't Stop Talking}}. Both of these books seek to recontextualize behaviors of
sensitive people that are often portrayed in a negative light. These behaviors
are presented in a way that showcase a sensitive person's strengths, rather
than as weaknesses. I've found them extremely helpful as I find my way through the world.

\hypertarget{on-craft}{%
\chapter{On Craft}\label{on-craft}}

I think that most starting artists have an overly romantic view of craft. They buy into the cinematic notion that the conception and creation of an artistic work is a smooth process that proceeds like a feverish dream from the moment of inspiration in a single session. This makes for good film, but it's really not how the majority of artistic pieces are made.

The truth is that good works of art take a lot of work. Much of that work is good old fashioned problem solving. Many of us think that once we have the idea or the inspiration, the work is nearly done. But real creative work is dealing with unanticipated issues. For example, in translating a sketch of an outfit to an actual piece of clothing, there are tons of unanticipated problems, such as how two pieces of cloth must come together to fit correctly on a real person. Musically, we may have two themes we want to work together, but finding the chords or connecting melodies to do this requires a lot of experimentation. Writing a good story or a novel requires problem solving in executing a plot correctly, finding actions that are true to the characters and their motivations that execute the plot. A good poem is like a machine, with no word out of place and each word contributing to the effect of the whole poem. Executing each of these art forms well requires creative problem solving.

I propose that you divide creating art into three distinct activities: 1) \emph{capturing inspiration}, 2) \emph{working/problem solving}, and 3) \emph{editing}. The majority of your work will be working and editing. Note that working and editing must be distinct processes. When you work, you must be free and uninhibited; when you edit, you must be merciless in what you cut and what you include. If you do the two together, you will end up being creatively paralyzed.

\hypertarget{capturing-inspiration}{%
\section{Capturing Inspiration}\label{capturing-inspiration}}

First of all, toss that notion of working on a piece in a single feverish session away. You live in the real world, and barring having a wealthy sponsor, working like that is a luxury most of us don't have. You need to be able to capture your inspiration when it comes, as it often does at inopportune moments, such as the shower, on a walk, or during work. Lots of tools exist for capturing these moments of inspiration, such as notepads, recording apps, and cameras. You should get over any feelings of awkwardness or self-consciousness when using them in public.

Your capture tools should be fast, simple, accessible, and adequate enough to capture your ideas accurately. Transfer your ideas to some sort of long-term storage. Store your ideas in a notebook, a file folder of hummed recordings on your desktop, or an photo album of beautiful pictures. The point is to be able to rapidly review ideas for follow up. Be sure to eventually act on your inspiration. Otherwise, those ideas lie fallow, much like rolls of exposed but undeveloped film. It's wasted potential if you don't act on your ideas, and your muse will be displeased. You may find your wellspring of ideas may dry up because you failed to act on them.

You may find it useful to dispose of ideas that have sat in long-term storage for too long. These unworked ideas can weigh you down creatively and should be let go. Occasionally revisiting and clearing out your backlog is essential to the creative process.

\hypertarget{the-work}{%
\section{The Work}\label{the-work}}

The next step is working, the creative problem solving. You act on that inspiration that you've captured, when you have the proper time to work at it. You must be willing to do the work. There are no shortcuts. You must learn what works and what doesn't in the context of your piece. You must not be precious about the work. Working only when you are ``inspired'' is a specious notion. The most important thing is to start and maintain momentum. Studying theory (music theory, color theory) and art history can help you understand what solutions to similar problems others have used. But your solutions must be practical and personalized to your vision. Don't be afraid of making mistakes. Digital tools give you the freedom to go back if we make a mistake. Make multiple versions if you are unsure.

Working is far from a linear process. I rarely work from the start of a piece to the end of it in a linear fashion. My own musical process tends to be to create tons of versions of a musical phrase and then figure out how to connect them together. I need to have a seed idea, which can be a tiny repeated sample or theme to work on. When I write, I often put all of my ideas on the page at once, using a mindmap. Seeing all the ideas on the page at once helps me discover the structure of the piece I write and allows for happy accidents to shape the piece.

You should work without feeling self conscious and critical. Save being critical for later when you have had the proper time to see the shape of the work. You can always go back and fix things later.

A note about those who improvise to produce their pieces. This is a unique process unto itself, and it's easy to generate a lot of material using this process. If you have the improvisational skills, it's relatively easy to hit record. Where I have seen many an improvisational musician fall down is in the editing stage. Due to the ephemeral nature of performance, what works live does not necessarily work in the recorded setting, where the listener can endlessly relisten to something (I find many live recordings to be tedious because of this). Selecting the moments of improv that work and stitching them together can be a difficult process, and it can take much longer than the actual improv. When Marcus Fischer and I put together \emph{Tessellations}, the base of it was improvised, but then we had to mercilessly edit 20 minute sessions into coherent songs.

\hypertarget{editing}{%
\section{Editing}\label{editing}}

When you have done some work, then you are ready to edit. Make sure to separate editing from working, otherwise it will paralyze your working process. It is often good to give yourself some time between working and editing. You can edit the work with a fresh perspective and with less emotional attachment to ideas.

During editing, you must understand how each individual idea contributes to the piece as a whole. Be merciless and don't get emotionally attached to any one idea. If an idea works, keep it; if not, save the idea in another file of other ideas to pursue later. If it's a good idea, it will eventually find its place. If not, just know that the creative road to every standup routine, novel, and symphony is littered with lots of discarded ideas. Look at Beethoven's drafts; he discarded huge sections of pieces from draft to draft, and crossed out whole pages the manuscript. I personally find it comforting that all artists struggle. The ones who say they don't are probably lying.

The act of editing can be creatively cathartic in itself. Often, it's only by discarding ideas that I can see the true shape of the work. Once you have a vision of the form your work is supposed to take, the process will become easier. You will become more sure of whether an idea belongs or not as you continue.

Oftentimes, the editing process generates new ideas, which may be useful to the work or external to the work. Be sure to capture these and use them in the next working phase.

\hypertarget{the-thick-of-it}{%
\section{The Thick of It}\label{the-thick-of-it}}

You will often iterate between working and editing. They take two different states of mind, so it is often useful to schedule a working period with one or the other in mind, based on your current mindset. When you're in the thick of it, it can often seem like an endless process. Fight the feelings of inferiority that arise. Persist, and you'll see it through.

Hopefully, through this process, you will discover the structure, or form, that suits your piece best. Do not dismiss finding the proper form; it's the architecture and the scaffolding that you hang your ideas on. A good form can be the difference between someone being indifferent to your piece or making a deep and lasting connection. Think of a good pop song; it gets to the point and it does it in a structure that can still surprise us to the day. Presentation is important.

When is a piece done? When viewed externally, other artists' work might seem to have an air of seeming inevitability to it. In interviews, these artists often play up this sense of inevitability, suggesting they are merely tools of fate. This is mostly hogwash. A work of art is really only finished in the mind of its audience. Mark Snow, in an interview with Kumail Naijani, remarked at the indifference Chris Carter and the TV executives had to the X-Files theme before the series aired; it was only after the show aired and the audience liked it that they acknowledged it as a masterpiece. One of my most used pieces, ``Silhouettes'', was originally twice its length; it was only after I had to fit it into a time requirement that the ``right'' length emerged. My point is to realize that the endless tinkering has to end at some point.

However, you should feel good about the work for it to feel finished. If you don't feel it's done, then it probably isn't. There will always be nagging doubts about your work; if you are worried, release your work in a safe forum, send it to artistic friends you trust first. It's often great to get the perspective of an artist outside your field; they will talk about it in artistic language, but unhindered by the perspective of your field. Make sure to extend this courtesy to them when they ask for it as well.

\hypertarget{having-a-process-to-creatively-enable-you}{%
\section{Having a Process to Creatively Enable You}\label{having-a-process-to-creatively-enable-you}}

I realize that most of you at this point are probably dismissing me for taking the romance out of the creative process. These thoughts about process and craft are hard won after years of creatively floundering about. I realized that the way I worked was counter productive, and wasn't any more fun. By separating working and editing, my process has become way more fun and less dramatic. Starting a work has become an exploration of possibilities rather than one burdened with unreasonable expectations. I give myself permission to utilize happy accidents and to abandon projects that don't inspire me.

The point really is to start, not to regret that you never did any writing, composing, or painting late in life. Artists with staying power find that the work itself is incredibly rewarding; everything else is really icing on the cake.

\hypertarget{resources-1}{%
\section{Resources}\label{resources-1}}

This piece is the distillation of my own personal experience and observing the creative processes of others. As a corrective to the cinematic notion of craft, here are some good movies about the work involved in the creative process.

\href{http://m.imdb.com/title/tt3204392/}{The Kingdom of Dreams and Madness} about Hayao Miyazaki and Studio Ghibli is a powerful meditation on a master of animation and how he works. I thought it especially interesting in light of how his studio has adapted to his process.

\href{http://m.imdb.com/title/tt0109508/}{Crumb}, about the work of R. Crumb, was a big influence in getting me to understand how important working is to creativity.

\href{http://m.imdb.com/title/tt0424509/}{Touch the Sound}, about deaf percussionist Evelyn Glennie, is a profound look at being inspired by the world and her creative process. There is a great moment in the movie where she and Fred Frith are banging on pipes in a warehouse where she is recording.

\href{http://m.imdb.com/title/tt0151568/}{Topsy-Turvy} is a wonderful view of the amount of work it takes to mount a musical (Gilbert and Sullivan's \emph{Mikado}) from conception to the stage.

\hypertarget{on-recharging}{%
\chapter{On Recharging}\label{on-recharging}}

You are not a factory. If you are commercially successful, the treadmill of commercialism may demand you churn work after work, piece after piece. But tailoring your work to the external world comes at the cost of diminishing your internal world. Creating art is not manufacturing. It is not a commodity, it is a gift to the world. There is a tricky balance in commodititzing your work and letting the work remain true to itself. I've seen many successful artists burn out because they were not able to say no to the flood of commercial expectations and job offers. These artists burn out because they don't take the time to recharge their creative batteries. New ideas don't come because they have become complacent and they have not appreciated the source of their creativity.

How do you recharge? First, get off the treadmill. Schedule time to recharge. You should view recharging as a necessary part of your work. It may be difficult to justify to yourself, but think of it as an insurance policy for your creativity. If you don't protect your creativity, no one else will.

In order to recharge, there are many activities you can do to jar yourself out of creative complacency. Here are a few I've found to be good. I'll talk about each in turn.

\begin{itemize}
\tightlist
\item
  Revisiting works that inspired you previously.
\item
  Walking or interacting with the natural world.
\item
  Trying something new or putting yourself in a new mental space.
\item
  Working in a different art form.
\item
  Mindfulness training and meditation.
\item
  Avoiding social media.
\item
  Being less hard on yourself.
\end{itemize}

Revisit your creative source and honor your muse. Sometimes this means revisiting the distant past. I think about re-listening to albums that I was obsessed with when I was younger and try to approach these albums with fresh ears. Why was I obsessed with them? I draw energy from remembering my old enthusiasms.

I draw energy from walking in nature. I am always inspired by the infinite variability and adaptability of life, of how plants grow and develop. The forms of nature are always energizing and inspiring to me. Lying in a hammock, watching the leaves tremble in the wind as I sway back and forth, creating delicate rhythms and textures that I find inspriring. Bird calls embed themselves into my subconscious.

I also recharge by trying new things, and listening to new music. I love hearing music from different parts of the world, from Javanese Gamelan, to Scandanavian drone metal, to Ethiopian Rock. Trying new things puts you in a different frame of mind and lets you see things anew. Pay attention to your feelings when you try new things. Are you uncomfortable? Are you energized? Harness those feelings, remember them. I think that feeling of discomfort and uncertainty is important to creating new art. When I start getting reductive about my music, I am in a bad space with it.

Go to shows, support others in art. You never know when you will be inspired by other people's work. Sometimes the act of connecting with new people can be energizing. Listen to new music with friends, and talk about it. What worked and what didn't work? Can you build on that? Does your dislike of a piece spur you on to do better?

Try mindfulness training, such as meditation. Again, the goal is to see things differently than your current viewpoint. Maybe you've been beating yourself up over a failed project to the point that it's impeding your current work. Mindfulness training can help you step outside of yourself and gain clarity on issues.

If you have commercialized one of your art forms, it can be sometimes difficult to deal with the crushing weight of audience expectations. I have found it useful to have another parallel art form (in my case, photography) that I do for myself. A surprising number of actors are also painters. I draw energy from rediscovering the beauty of the world through my photography. My photographs serve as reminders to myself that there is beauty in the world if I dig deep enough.

Part of recharging is unplugging from things that emotionally drain you, especially social networks. When I am burned out, I find social networks emotionally and creatively draining. Everyone's endless updates make their lives seem much better than mine, and it drags me down into a depressive hole. If you feel this way, you should unplug (thanks to Tim Gray for this suggestion) and focus on real life interactions. When you feel worthless is when you most need to interact with real life people. Being with people reminds you that everyone is not perfect, and that everyone struggles.

It's also worth reminding you not to be too self critical or harsh on yourself if you do not meet your own (possibly unrealistic) expectations. Beating yourself up might be an ok short term strategy, but we are playing the long game. The desire for perfectionism can be the worst creative block of all.

So, in conclusion, to creatively recharge you need to jar yourself out of complacency and go where you are uncomfortable. You need a new concept and perspective of what your art can be and how it fits into the rest of the artistic world. You need to revisit your creative source and honor it.

\hypertarget{resources-2}{%
\section{Resources}\label{resources-2}}

\href{http://www.lewishyde.com/publications/the-gift}{\emph{The Gift}} by Lewis Hyde is a great read and a reminder of why we make art. Making art is making a gift to the world, and it's ultimately more important that you gave the gift to the world rather than how it's received.

\href{http://lifehacker.com/how-you-can-learn-to-finally-really-relax-1548045887}{\emph{How to Really Relax}} is a nice reminder of things you can do to recover from burnout.

\hypertarget{on-depression}{%
\chapter{On Depression}\label{on-depression}}

I have a depressive personality. When I am depressed, my brain actively works against me. I have feelings of worthlessness and I doubt myself. I obsess and ruminate about the few bad things that happened that day, way out of proportion to their effects. I play the comparison game, comparing myself to others. I always lose the comparison game.

I have found cognitive behaviorial therapy (CBT) to be one of the few really effective treatments for my depression. I've gotten much better at self-soothing, comforting myself when I am in a depressive downwards spiral. I have a lot of self-defeating patterns that I fall into, and which have their own terrible power over me. I ruminate and brood, which makes me feel like I am inferior to other people and makes me overinterpret the thoughtlessness of others way too personally. (In fact, I think you could call me \emph{the boy who took things too personally}.) CBT has helped me dispel the power these patterns have had over me. I am, in fact, as good as everyone around me. I now choose to be bemused rather than openly irritated at thoughtless behavior. I feel like my agitation and anxiety have lessened to the point where I feel less overwhelmed.

CBT has forced me to re-evaluate what I consider the worst case. For example, what is the worst thing that would happen to me if my bag broke open on the sidewalk? One of my big triggers is when I feel I am being ignored, which can lead to self-destructive and alienating behaviors. It's during these times when I have the overwhelming feeling that ``everyone is a jerk'' - something I still struggle with. But I now know that it is a perception in the moment and not the actual reality.

The rumination can be the worst part of depression for me. I can get stuck in endlessly replaying awkward social interactions and beating myself up for it. This rumination can make me feel like I'm worthless, my art is worthless, and \emph{no one wants to hang with me}. I get caught up in the thoughts that I am not reaching the goals I set out for myself. What helps me stop this endless cycle is mindfulness and meditation; it helps me get out of myself and this thought process. The art itself can also be therapeutic and cathartic, but only if I let go of my expectations about it. I've also learned to be less harsh on myself. I may get less done than I'd like to, but I'm also more mentally healthy.

CBT has also helped me recontextualize what I originally thought of limitations. I used to think I was a total introvert, which to me meant that I wasn't good at interacting with others. To the contrary, I am quite good at interacting with people when I have a role, such as teacher or mentor. I know what to say to put people at ease. People tend to interpret my lower energy state as authoritative and calming, which makes my own nature work for me, rather than against me. I've always had grace under fire, and I handle stressful situations external to me better than others, and that is partly due to my depression. Teaching helps me and energizes me.

After all these years, I feel like I finally understand Michael Caine's advice to ``\href{https://books.google.com/books?id=ES6ZAAAAQBAJ\&pg=PT91\&lpg=PT91\&dq=use+the+disadvantage+caine\&source=bl\&ots=0NxeOR-4iO\&sig=XsFL4clh_687Cr3UN-nZZ_s_bIs\&hl=en\&sa=X\&ei=9XhwVcb-H8nZoATFwoOABA\&ved=0CCAQ6AEwAA\#v=onepage\&q=use\%20the\%20disadvantage\%20caine\&f=false}{use the disadvantage}''. That is, use what people view as flaws to your advantage. I used to criticize myself for not being more extroverted and outgoing. I have instead learned to use my own curiosity and empathy for others to connect with them at my own comfort level. I used to be anxious about embarrassing myself, and be anxious about saying the ``wrong thing''. I've learned since that making mistakes is human, and can actually connect you to other people. I used to think my software and music had to be perfect before I released it. I've since learned to release early and welcome feedback. My sensitivity, which I often considered a weakness, is actually one of my greatest strengths. I am a good analyst because I am data sensitive, sensitive to patterns in data. Likewise, such sensitivity is part of {[}what makes me an artist{]}(\{\{\textless{} relref ``2015-05-09-on-sensitivity.md'' \textgreater\}\}).

These judgements about myself did not come from nowhere - they are derived from reactions of others to me that I wrongly internalized - and I would plead to those who are more neurotypical to not spread judgements of others based on an initial reaction of depressive people. Seeing people grow up in the age of social media worries me. Words can hurt when you are sensitive and depressed. Think about that the next time you get outraged or are feeling snarky.

Believe it or not, there is also a positive side to my rumination - if I focus it on the right things and don't get obsessive. I process feedback and learn from my mistakes. I show dramatic and rapid improvement in learning new skills because of this. I just try not to focus on details that are in the past and which I cannot fix. That can be an uphill battle.

In conclusion, my depression is part of me whether I like it or not. I finally feel that I work with it now rather than against it. It has been a very slow process of reducing my expectations of myself and accepting my true strengths. I refuse to believe that good art comes from personal drama; I think it really comes from curiosity and compassion. I hope that this essay helps someone struggling with similar thoughts.

\hypertarget{resources-3}{%
\section{Resources}\label{resources-3}}

The most important step if you are depressed is to seek help. You will need to unpack your emotional baggage in order to unburden yourself, and that is best done with a therapist. \href{http://www.adaa.org/finding-help}{The Anxiety and Depression Association of America} is a good place to start. Keep in mind that CBT doesn't work for everyone and treatment can be an uphill battle. A newer type of therapy, Dialectical Behavior Therapy (DBT), may be a better fit for you. Medication may be an important part of your treatment plan - and finding the ones that work can be a long process of trial and error.

Lars von Trier's \emph{\href{http://m.imdb.com/title/tt1527186/}{Melancholia}} is an important movie for me in that it helped realize that there is grace in depression, and that depression can carry a quiet authority. Neurotypicals are not equipped to handle crises as well as depressed people.

Every couple of years I revisit Woody Allen's \emph{\href{http://m.imdb.com/title/tt0094663}{Another Woman}}, a quietly devastating movie about unpacking your demons. I love Gena Rowlands as a philosophy professor whose self deception can only take her so far. Soon enough, regrets and repressed feelings must be given their due.

\hypertarget{on-success-or-how-to-suck-it-up-and-be-happy-for-your-successful-friends.}{%
\chapter{On Success (or, How to suck it up and be happy for your successful friends.)}\label{on-success-or-how-to-suck-it-up-and-be-happy-for-your-successful-friends.}}

It can be egotistically bruising when one of our friends is successful and by comparison, we are not. But you should be happy for your friends when they are successful for the following reasons. I'll talk about each in turn.

\begin{itemize}
\tightlist
\item
  \emph{Success in the art world is rare, ephemeral, and should be celebrated.}
\item
  \emph{Their success is partially due to you and your artistic community.}
\item
  \emph{Bitterness destroys no one but you}.
\item
  \emph{Comparing yourself to others is a losing game.}
\end{itemize}

The world itself is hostile, or at best, indifferent to art. As artists, we need scenes to support us and our visions when the rest of the world does not. There are countless scenes that are examples of this: I think of the Beats, the Bloomsbury group, the shoegazers, the NY Downtown scene that gave rise to Talking Heads, Patti Smith, and Television. Artists in each of these groups supported each other, helped each other, attended each other's events despite their widely disparate visions of art. Such creative relationships can energize the work, as David Bryne talks about in his chapter on CBGB, \href{http://loud-time.blogspot.com/2012/12/this-aint-no-memoir-how-to-make-scene.html?m=1}{``How to Make a Scene''}.

Success itself in the art world is rare and should be celebrated. I believe that we as artists should shed the notion of competing with each other and instead encourage each other to grow and develop. Artistic competition is an illusion perpetuated by the capitalist market and the small pot of grant and foundation money that exists for artists. When the success of an artist is the product of a scene, that success belongs to the whole scene. Success itself may be ephemeral, but a musical community is not.

Success is partly the product of hard work and partly the product of uncontrollable forces, which most people call luck. I prefer to think of it as an alignment of broad cultural priorities with the work of the artist. Most artists, though they work hard and have good promotion, fail to `make it' - that is, fail to have their work recognized by a larger audience. For a select few, their work will resonate with a larger audience, but it seems like the duration of this resonance is getting shorter and shorter. Momus, the clever quipper that he is, riffed on Andy Wharhol's statement that ``everyone will be famous for 15 minutes'', instead asserting that ``everyone will be famous for 15 people''.

You should also be happy for the successes of others because it's better than the alternative, which is being bitter. I think bitterness is the enemy of art. I think we all know someone whose bitterness has partially ruined their life, and that has tainted the lives of those surround them. Bitterness ruins relationships and leads to loneliness. I would rather have the support of close artist friends than the vague support of a larger audience.

It's equally important to manage expectations. Unrealistic expectations can doom us to feelings of failure, especially when those expectations arise from comparing ourselves to the success of others. Don't compare yourself to others and beat yourself up. This is the way of bitterness. Complaining that you don't have the recognition you ``deserve'' will only gain you a reputation as a complainer. It's better to find your own path if the current paths aren't working for you. Countless artists have struck out on their own when the current art world had nothing to offer them. It can be a hard, but rewarding path if you take it.

It took me a long time to realize that external validation is largely useless to me. What should matter most is that you feel good about releasing the work into the world, that you executed it the best of your abilities, and made compromises that you can live with. Integrity is what matters the most to me.

For those who are successful in this moment, I would remind you of what Joan Rivers says in an episode of Louie: remember the people who helped you on the way up, because you'll need them when you're on the way down. Likewise, entitled musicians who think they deserve deluxe treatment and treat everyone else badly probably aren't going to be tolerated for long. Those who define their success by the number people they think they can treat badly are doomed to be alone.

So, in summary, success can be fleeting and is not usually created in a vacuum, it's usually a product of an artistic community. In this context, we should celebrate the success of our friends. We should also avoid comparing ourselves to others and becoming bitter as a result. To do so is to practice gratitude for your audience and to cement your position in a musical community. Given that success is not handed to all of us, I am very happy with community and friends as a consolation prize.

\hypertarget{on-artistic-influences}{%
\chapter{On Artistic Influences}\label{on-artistic-influences}}

It's often said that ``good artists borrow, but that great artists steal.'' Stealing is a necessary process to making art, but I think it only works when you have a thought process behind it. What you steal should be processed by you such that the influence is unrecognizable on immediate listens. It's not enough to reference an era; you have to make that era yours. Otherwise, you are simply beholden to the past and are contributing nothing new to the world.

By all means, you should consume and be inspired by other people's art, but you must destroy them and surpass them as well. Killing your idols is necessary to progress as an original artist. In order to progress, you must develop your own aesthetic. An aesthetic is not a brand; an aesthetic is a way of perceiving and interpreting the world. It is constantly mutating in response to the world, but always has a core.

What is an aesthetic? One definition is that an aesthetic is ``a set of principles underlying and guiding the work of a particular artist or artistic movement.'' Your own aesthetic must be deep, or it will not sustain you. I've seen many artists who believe their `sound' or personal stamp is derived from a single technique, such as a single effect plugin, or a color palette. Relying on a single tool or influence to shape your art is unwise. You must have a creative process that is external to your tools, or your tools will determine your creative process. Without an external process, you will reach creative dead ends. Developing this process is part of discovering your own aesthetic, and discovering it can be a tortuous process. But it can also incredibly fun and never ending.

An aesthetic is not a brand. A brand exists to give identity to merchandise, and an aesthetic is much more than that. It's your through line that when understood, will unify your body of work. It defines what your work is, but also what it isn't. It should reflect your enjoyment, but also your frustrations with current artists.

You should protect your aesthetic, as it's what makes your work unique. If you are lucky enough to inspire copycats, you must go beyond them into realms where they would not dare. People may copy your surface style, but they can't copy your aesthetic if it is a deep one. If you can, it's probably better off not talking about it in public. I think it's better for people to discover it for themselves. I think about how mysterious those Boards of Canada records were when they first came out. Everything about them was a fascinating mystery, from the voices, to the song titles, to the 70's warbly synth soundtrack feel.

Get good at taking works you admire apart. Understand why a particular musical moment is powerful and moving. Is it because they broke an established pattern? How did the artist overwhelm you and subvert your expectations? Was it a shock, or was it a more pleasant surprise? How did they do that? These details are all grist for the creative mill.

It's always interesting to find out about what techniques an artist used to get a particular effect or sound, but I usually find that obsessively collecting gear to reproduce a particular sound is counterproductive. It's a scavenger hunt whose results are always mildly disappointing to everyone. Even if I had every bit of Kevin Shields' gear, I doubt I would sound like him. Thinking about his technical limitations and how he overcame them is much more inspirational to me. Thinking about the creative principles behind a work is inspiring as well. I use a lot of detuning in my work, inspired by My Bloody Valentine, but it's manifested as cello overdubs rather than trying to recreate it with detuned guitars. Find the higher order principles that make a piece work and you can make them your own.

There is nothing new under the sun. Most creative techniques have been used at one point or other, but they haven't been used by you. What ultimately makes a creative work unique is you - and your mix of influences and artistic ambitions. Cultivating your own unique aesthetic is an important part of the artistic journey and will serve you in good stead.

\hypertarget{resources-4}{%
\section{Resources}\label{resources-4}}

\href{http://gapingvoid.com/2004/10/18/how-to-be-creative-in-pdf-format/}{How to be creative}. Early on, I found these principles very helpful, since they are focused on making your work unique.

\href{http://lifehacker.com/demystifying-the-muse-five-creativity-myths-you-should-1688503554}{Myths about creativity} this was a nice little article about busting some untrue myths about creativity.

\hypertarget{on-musical-collaboration}{%
\chapter{On {[}Musical{]} Collaboration}\label{on-musical-collaboration}}

I love to collaborate. One of the reasons I enjoy playing the cello so much is that it can play many musical roles: lead, rhythm, backup, counterpoint. Having the cello as my instrument forces me to be versatile, and to find my sonic niche in every collaboration I participate in. However, not all collaborations turn out well; some fizzle after the first piece we work on. Usually after the first piece we work on I will know whether my partner is a good collaborator for me.

For me, a good collaborator is open to new ideas. They are believers in the \href{https://en.m.wikipedia.org/wiki/\%22Yes,_And\%22_rule}{``yes, and\ldots{}''} school of improv, building upon your work rather than tearing it down. Collaboration requires a fundamental level of trust, and getting to this point of trust is not easy. A lot of people seem to think that collaboration is a competition, wanting to stand out from others they work with. I find these kinds of collaborations to be shallow and draining.

For me, the best collaborations enable us to jointly venture into unknown and uncertain territories. My collaborator and I may not know the boundaries of what we do, but we are open to exploring. We may start with a simple idea, or with improvisation. But our goal is that we discover the form and structure of the piece together. We edit together, bounce ideas off of each other, and decide when a piece is finished together. We make mistakes together, and may find that our best ideas arise from these mistakes. We make a safe space for each other to be musically vulnerable. Only when that respect is there can the true music making begin.

Part of being a good collaborator is making a safe space for your collaborators to try new ideas. Make it safe to make mistakes together, since it is often the mistakes that become the foundation for something interesting. If you are dismissive of my ideas or grab the keyboard every time I try something, that is not making a safe space. Communication is key. Make any criticisms constructive rather than destructive. But be honest.

All artists have big egos, but we must set them aside when we collaborate. We cannot be threatened when our collaborator plays better than us (our collaborator may have to realize that he is grandstanding). We cannot rewrite or rerecord our collaborator's parts, as much as we would like to. We must be willing to give up a measure of control in the collaboration.

By giving up control, we give the collaboration room to breathe, develop, and grow. We may find our collaboration sounds completely unlike our separate work, but only if we are open to this possibility. You don't have veto power over your collaborator's contributions, nor they over you. Yes, that may not be the way you would have executed a part. But that's why you collaborate, to get out of your head and to work with ideas which aren't your own. Some people are so in their own heads that their own aesthetic and ideas don't play well with others. You must always be aware of overdominating the musical conversation. If you are inflexible over a piece, you need to let that go.

I am far from an ideal collaborator, but I attempt to give my collaborators their space. I make sure there is space for me to react to their contribution and vice versa. Many times, it takes a lot of experimentation and improvisation to discover the form of the music together. Often, the first moments of an improvisation are uncertain. New ideas and combinations are tried, and they often fail. Discovering the common scales and modes between you can be difficult, if such things are usually not discussed beforehand. The improv and its development is much like walking a tightrope.

All of these interactions become more complicated when more people are added to the collaboration. A two person collaboration is mostly a conversation, whereas a three person collaboration can become an uneven power struggle, a la Sartre's \emph{No Exit}, a work that posits that ``hell is other people''. The minority opinion in these cases must be respected, and not dismissed. Again, it's not a competition of ideas. Respectful discussion is encouraged; don't look at any decision as a mutually exclusive one. Often, compromise can lead to a new path being discovered.

Lastly, part of the secret to a good collaboration is saying no to the wrong collaborators. Seriously, don't work with jerks, no matter what they promise. A jerk, by my definition, is more interested in the outcome than the process.

A collaboration can be a wonderful side road to explore, producing new and unexpected work, and it can be a necessary creative jolt. But a true openness of spirit is required to make it work. Disregarding personal dynamics and feelings in collaboration leads to poor work, and is often at the root of many band breakups. However, if you lose the ego, you may find a brand new source of creativity.

\hypertarget{resources-5}{%
\section{Resources}\label{resources-5}}

\begin{itemize}
\tightlist
\item
  Read Aristotle's chapter in the \emph{Nicomachean Ethics} on virtuous friendship for \href{http://cantory.blogspot.com/2007/12/aristotle-and-his-view-of-friendship.html}{inspiration on how to be a good collaborator}.
\item
  \href{http://www.goodreads.com/book/show/18050769-high-status-characters}{\emph{High Status Characters}}, the oral biography of the Upright Citizens Brigade, has an excellent chapter on the ``Yes, and\ldots{}'' school of long-form improvisation. The rules are useful musically as well.
\item
  \href{http://openendedgroup.com/images/CreativeCollaboration_OpenEndedGroup.pdf}{Rules for creative collaboration} is another formulation of these rules for collaboration. I don't particularly agree with all of them (I disagree about their feelings about compromise), but as an additional perspective, it's useful.
\end{itemize}

\hypertarget{on-theory}{%
\chapter{On Theory}\label{on-theory}}

Many artists have a bad relationship to theory. This is understandable. There are a lot of bad teachers of theory, who fail to add that theory is mostly a suggestion, and that figuring how to break these rules of music theory is part of the fun of learning it. Also, most of us were taught theory when we were adolescents, which made theory seem restrictive and something to rebel against.

Theory is really only the history of art distilled into the form of rules. Someone, at some point, attempted to separate the good art from the bad and building a set of rules seemed like the best way to do this. It is true that music is all about proportions, intervals, and math. But the reason why one piece affects us cannot be easily be quantified. As much as some art theorists believe, there are no single unfiying ideas about what separates `good' art from the rest. The entirety of music and art, much like language, resists easy classification. Theory only has the authority we assign to it.

There is a prideful sneer to certain musicians when they confess they are completely self taught, and they don't need music theory. They only need three chords, they say. Well, sad to say, but these people are still using theory, but in a very limited form.

On the other side are self indulgent theory junkies who put multiple key and signature changes in a song without bothering to ask whether there is actually intention and reason behind these choices. Their music often comes off as self indulgent. Their pieces are often crammed so full of ideas that none of them have room to develop or breathe.

There is a balance between these two approaches. I think that part of achieving that balance is realizing that theory can help us when we're stuck for direction.

Theory can take many shapes and is not just limited to what we learned about harmony in university. Here are six examples that may expand your concept of theory and its place in music:

\begin{enumerate}
\def\labelenumi{\arabic{enumi}.}
\tightlist
\item
  There are cross-pollinating theories, such as those suggested by Henry Cowell's \emph{New Musical Resources}.
\item
  Theory can include chord leading, which is a way to transform one chord into another.
\item
  Theory can include the study of tradition, such as the works of Bartok.
\item
  Theory includes general compositional strategies such as Brian Eno's Oblique Strategies.
\item
  Theory includes ways to incorporate randomness such as those of John Cage, or non-traditional collaborators such as Messaien and birdsong.
\item
  Theory also includes improvisational game pieces such as those of John Zorn.
\end{enumerate}

I will talk a little about each of these types of theory.

Henry Cowell's \emph{New Musical Resources} is a fascinating little book and I highly recommend it as an example of theory expanding your boundaries. I think it is a wonderful example of applying concepts from one part of music to another. In it, he proposed to apply the principles of the overtone series to rhythm. He suggests new cross-rhythms that could be played based on the harmonic series (a simple cross rhythm is to play in 2 while someone else plays in 3, Cowell suggests crazy cross rhythms such as 5:8 and 4:7). His theory of tone clusters uses major second intervals instead of major thirds as the basis of chords, producing a new type of tonality. In many ways, Cowell's approach is inspirational, especially in how it takes principles that govern one aspect of music and map it to another, producing new and unheard territory in the process.

You might have encountered chord leading as suggestions for selecting a set of chord changes as the structure of a piece. Chord leading suggests a number of chords that might follow another chord you have selected in your sequence. Chord leading is in effect a set of rules for transforming one chord into another in a musical sequence. We may need to only change one note in a chord to result into a completely different musical effect. Chord leading is pretty much a suggestion for building musical structures that are readily accepted by others. Shostakovich flaunted these rules. I think of his trio where in the context of his slow movement, the arrival of a major chord is a jarring event. But he could have only achieved such an effect by understanding chord leading and how to subvert our expectations as listeners.

Theory can ground us in a musical tradition by showing how our work relates to the past. I can't imagine what Bela Bartok's music would sound like without his intense study of Hungarian folk music and tradition. To participate in a tradition can add a powerful sense of familiarity to the work, especially if our piece uses some traditions and subverts others. (A very funny example of this is Erik Satie's \emph{Sonatine Bureaucratique}.) But we must provide enough of our own vision or risk the work being weighed down by tradition.

Theory can include compositional strategies, such as Brian Eno's \emph{Oblique Strategies}, a deck of cards with instructions which provide us with ways to recontexualize musical ideas. One of the most famous is ``Honor thy mistake as a hidden intention.'' These strategies can provide us with unexpected directions to take a musical piece, but their implementation must be organic and true to the nature of the piece. Otherwise, they are about useful as notes from studio executives.

A theory of chance and probability can be freeing. Both John Cage and Olivier Messaien incorporate such theories of randomness and chance into their work. This frees themselves of part of the compositional burden. John Cage's pieces all have random elements but, not all of the musical elements in his pieces rely on this randomness. For example, in \emph{Apartment house 1778}, Cage used chance to decide which parts to subtract from American Hymns. Randomness was used as a partner in composition, not as the overall determinant of the final work. Similarly, Messaien used birdsong to inject new ideas into his work, and his modes of limited transposition are a form of compositional restriction derived from analysis of his own work. However, even in the use of such systems, Cage and Messaien never abandon their own compositional tastes.

Theory can be embodied in those pieces that are improvisational games, such as John Zorn's ``game pieces''. ``Cobra'' is one such musical game, consisting of rules for improvisation that are switched on the fly by what Zorn calls a ``prompter'', who functions much like a musical traffic cop, directing the musical flow from one improviser to the next. The theory in these games is a theory of musical interaction meant to increase the intersestingness of the piece to the listener. The games themselves are types of power struggles and with the right musicians, these power struggles can be compelling dramas.

Theory used in the context of connecting us to tradition and helping us overcome our ruts can be liberating. It's only when we think of theory as rigid rules in a system that we find it restrictive. Yes, there was originally a rule of ``no parallel fifths'' in baroque composition. But power chords came out of flaunting this rule. Total serialism was often used as a rigid system and as a result, many serialist pieces lost the human element by giving up too much compositional control. Such rigid compositional systems can be intellectually intoxicating, but are really dead ends.

No theoretical system can completely relieve the composer of the most essential quality needed in composition: intent. Your decisions must be purposeful, compelling, and organic to the nature of your piece for your music to be compelling to others. You must discover the integrity of the piece and fight for that. Otherwise, you might as well as leave your music to the machines.

I hope that this essay encourages you to think of theory in a new light. We've surveyed the field of music theory and realized that it encompasses many techniques and ways of thinking that can help us when we are stuck in a compositional rut. So, start exploring the wide world of theory. You might be surprised.

\hypertarget{resources-and-references}{%
\section{Resources and References}\label{resources-and-references}}

There are a ton of music theory books that push the envelope. Here are a few that I used to write this essay.

\href{http://www.therestisnoise.com}{The Rest is Noise}. If you want an introduction to composition in the 20th and 21st century, I can't think of a better book. Pithily written, it's a dishy account of a fascinating subject and a nice reference.

\href{http://zztt.org/lmc2_files/Cowell_New_Musical_Resources.pdf}{New Musical Resources}. Henry Cowell's slim volume provides us with an approach for recontextualizing music and its many aspects.

\href{http://www.musicarrangerspage.com/4606/chords-under-a-melody-chord-leading/}{Chord leading}. Here's a relatively easy to understand page that explains the concepts behind chord leading.

\href{http://www.rtqe.net/ObliqueStrategies/}{Oblique Strategies}. Brian Eno's deck of cards with compositional strategies have now taken many forms, including an app. These compositional heurisitics may be helpful. Now in its 5th edition.

\href{https://www2.ak.tu-berlin.de/~gastprof/collins/experimental-music/Zorn/americanmusic.28.1.0044.pdf}{Some Notes on John Zorn's Cobra}. Great essay analyzing the game piece ``Cobra''.

\href{http://rosewhitemusic.com/piano/writings/introduction-music-john-cage/}{The Music of John Cage} A book on the music of John Cage. I used some notes from the introduction in writing this essay. I also recommend his first book called ``\href{https://en.wikipedia.org/wiki/Silence:_Lectures_and_Writings}{Silence: Lectures and Writings}'' if you want to learn more about his ideas. Also check out his \href{http://www.ubu.com/sound/cage.html}{UbuWeb Page}.

\href{http://monoskop.org/images/5/50/Messiaen_Olivier_The_Technique_of_My_Musical_Language.pdf}{The Technique of my Musical Language} Olivier Messaien's master work on the techniques he used to compose. It's not an easy read, but it's very interesting all the same.

\href{http://www.ubu.com/papers/}{UbuWeb's Papers} house many classics of contemporary music theory, from Luigi Russolo's ``\href{http://www.ubu.com/papers/russolo.html}{The Art of Noises}'' and Pierre Boulez's ``\href{http://www.ubu.com/papers/Boulez-Schoenberg+Is+Dead.pdf}{Schoenberg is Dead}''.

\hypertarget{on-listening}{%
\chapter{On Listening}\label{on-listening}}

\begin{quote}
Note: I was asked by Steve Ashby, who teaches a music appreciation class at Virginia Commonwealth, to contribute my thoughts on listening to music as part of his series \href{http://rampages.us/mhis243/listeners-on-listening/ted-laderas/}{``Listeners on Listening''}. I've reproduced the interview below.
\end{quote}

\emph{You buy a new album, or hear a new piece for the first time, describe your routine/experience of its first listening.}

On first listen, I hope to be captivated and swept away, so I don't pay attention to details. I listen with my heart, and not my mind. I focus on moments in the music, where the music changes dramatically and it sweeps me with it. If I'm lucky, I might gain some knowledge of the structure, or form, and understand how it supports these moments.

\emph{On subsequent listens to that same record/piece, which aspects of the music do you focus your listening on. How does your listening perspective change over time?}

I try to keep my listening on a largely uncritical level. Listening truly critically can hamper my enjoyment of a piece. It's only later when I want to know how a piece works that I start to listen to it critically. Even then, it's impossible for me to completely take a piece apart. I tend to focus on single parts and how they contribute to the texture of the piece. How does the mix emphasize the melody or the harmony? How does the composer handle the transitions into those sublime moments? If I'm receptive enough, I can learn all of this and apply these concepts to my own work.

\emph{If you could choose your favorite listening environment, what would it be? what draws you to that place to hear the music you're listening too?}

It really depends on what kind of music I am listening to. One of my favorite listening experiences ever was seeing Tim Hecker as part of Rafael Irrizari's \emph{Substrata} Festival series. The music was overwhelming and the bass hit me in the guts. It was an incredible listening experience where I got swept up into the tidal wave of sound. I've only had that kind of visceral listening experience a few times, but they were magical moments.

Otherwise, I just love listening to music on headphones while I am doing nothing, preferably lying down. There's something about lying in a hammock, slowly swaying in the wind, watching clouds drift by while you are listening to music. Your brain expands and you get lost in the sonic world of the piece.

\begin{enumerate}
\def\labelenumi{\arabic{enumi}.}
\setcounter{enumi}{3}
\tightlist
\item
  \emph{How does one make their listening listened to? What is the best avenue to communicate your listening experience to others?}
\end{enumerate}

I think it is probably best to lead by example in terms of listening. I do love that listening parties are a thing nowadays. I think it's great that listeners can have a shared experience of listening to a brand-new recording.

As for sharing that experience, there are only a few people who I really feel comfortable sharing my private listening experiences with. These are people who know my tastes and I know theirs, so there is already common ground. There are countless ways to share music, but really only a few ways to share listening experiences with each other.

We must never forget that one of the functions of music is a social one. Before Beethoven, concert goers frequently talked throughout the concert! The notion of a rapt audience is a fairly new one in the history of music. Music in bars serves a different purpose than music in concerts, or at listening parties. People go to bars to forget about life, to commiserate, to hook up. So anything that doesn't fit in these categories is going to have difficulty being accepted at a bar.

\emph{How does your focused listening to music help you hear the world around you? How does it build awareness to the sounds in your everyday life?}

I am overly sensitive to the world - it is easy for me to get drawn into an overheard conversation on the bus or elsewhere. In loud social situations, I sometimes have to tune everyone out. Everyone seems so desperate to stand out in the world, and it can be an intolerable cacophony for me. Daily life can be painful without tuning it out. I have to find ways to dull this sensitivity.

Being in nature when I am sound sensitive is pleasurable in contrast. The rustle of leaves, creaking trees, the caw of crows, the sound of rushing water are very vivid to me. Going outside is my way of refreshing my ears, especially when I am in the midst of doing detailed musical work such as mixing.

\emph{Do you feel you listen to music differently as an audience member vs.~a performer? how so?}

Listening as a performer is very different than as an audience member. I incorporate a lot of looping and improvisation when I perform, so I must react to what I have layered before. When performing, I listen to understand which direction I want to take a piece in. I have to be open to the possibilities and I must give myself room to add to the looped performance. I often hear what I'm going to do in my head before I perform it.

When improvising with others, I listen for the holes in my band mates' performances to find my niche. The worst improvisers are the ones who don't listen. They steamroll over their bandmates' performances and are selfish with the spotlight. Improvising is not only about what notes you play, it's also about what notes you \emph{don't} play and how your part contributes to the whole gestalt. To play too many notes when you improvise is a sign of insecurity.

\hypertarget{undeveloped-film-or-the-tyranny-of-unfinished-work}{%
\chapter{Undeveloped Film, or the Tyranny of Unfinished Work}\label{undeveloped-film-or-the-tyranny-of-unfinished-work}}

Among other things, I am an avid photographer. I used to do a lot of black and white photography. I loved processing and printing my photos in the darkroom. One of the worst times in terms of my creativity happened because I lost access to a darkroom. I was shooting a lot of film, but then never developed it. This undeveloped film started weighing on me heavily, and not seeing it finished took a toll on my creativity.

I have to say that not developing this film made me feel the worst about myself and my art that I've ever felt. It's only now that I really understand why this was so.

Unfinished work is a mental burden and its presence can hinder your creativity. I know that an artistic work is rarely definitively finished, and \href{https://quoteinvestigator.com/2019/03/01/abandon/}{some say it is only abandoned}. Nevertheless, you must try to seek closure, even if that closure is somewhat unsatisifactory. Otherwise, the promise of these unfinished works takes up precious room in your creative capacity and memory.

When I'm composing, a looped bar of music can quickly become fixed in my mind and fail to become a song. In some respects, it can become a prison of my own making. Better to blow it up, make lots of versions of it very fast so it doesn't become fixed, and retrospectively make sense of it and discover how these multiple versions work together.

\hypertarget{dont-be-precious-about-your-art}{%
\section{Don't be precious about your art}\label{dont-be-precious-about-your-art}}

We fix creative ideas in our minds because we feel like they are precious, unrenewable resources and that to alter them would ruin them.

So, to get unstuck, we have to give up this preciousness. We must respect the idea, but not be bound to it. We have to disrespect it and not be precious about it. If we can maintain this balance, we will realize that creative ideas are indeed a renewable resource.

The only constant to art is that the rules are not fixed. Becoming fixated on what something \emph{should be} leads to being creatively stuck. Most of us will not conceive an art piece in full from the beginning. We have to be open to the journey of discovery and understand where that germ of an idea can take us. We need to explore the space of possibilities that germ can provide.

\hypertarget{free-your-ram}{%
\section{Free your RAM}\label{free-your-ram}}

I have a folder full of unfinished pieces. At some point, I will have to just not finish these pieces because they are weights on my creative brain.

Your creative brain has limited RAM. If you're fixated on the failures, and whether or not what you work on is going to be a failure, you leave little room for being creative.

What I do is that occasionally I will cull the list of things I am working on. I will render these songs as is and save them in my sonic graveyard. The chances I can recycle some of these ideas is high, and I am at least honoring their creation.

\hypertarget{get-it-out-there}{%
\section{Get it out there}\label{get-it-out-there}}

There is a case for half-formed things. The longer you have to dwell on something, the less likely you are to release it.

YouTube/SoundCloud/Medium/IG/Tiktok have all made it easy to get work out there quickly from the moment of creation. I think overall they have been good and have democratized art and who can make it. There is something good in releasing something that's created in the spontaneous moment, but there is also something good in working on something and putting effort into it. I'm trying to aim for a happy medium where creating is not wholly agonizing, but is rewarding in itself.

In a sea of endless creativity, our value lies in how we participate in the conversation of creation and how we react to other's work. You will grow faster if you attempt to dialogue with those who interact with your art. But you must not lose yourself in this conversation and only focus on what other people want. What makes you an artist is what \emph{you} want. Be true to yourself. But get it out there. Brian Eno and others have noted that a work of art has no value unless other people have experienced it.

You are on a journey as an artist. Oftentimes, you will stumble many times in finding your path. You must forgive your stumbles and celebrate your wins, no matter how small they are. Otherwise, the stumbles will loom large in your brain and you will be stuck. Make room for your wins in your brain.

\hypertarget{on-late-bloomers}{%
\chapter{On Late Bloomers}\label{on-late-bloomers}}

I am a late bloomer. It takes me a long time to figure things out creatively. My creative process almost never starts with a vision of what I want the work to be. I muddle my way through my creative process before I actually know what the work is about. I am not a visionary - I am a revisionary.

\href{http://www.newyorker.com/magazine/2008/10/20/late-bloomers-2}{Malcolm Gladwell's essay} suggests that early bloomer and late bloomer artists really do have distinct approaches to art. Early bloomers are conceptualists, whose idea of the work comes as a full formed vision, and creating the art is simply a matter of executing that vision. Prominent examples of conceptualists are Picasso and Orson Welles, whose best known works were executed in their youth (Welles directed ``Citizen Kane'' at the tender age of 23).

In contrast, late bloomers have an experimental approach, where each work of art is part of a continual process of exploration and improvement. When I work on a piece, it is often a long time from when I start a piece to the point where I understand the true form it will take. There are many artistic dead ends I must explore before I find the organic solution and path. Rarely is my artistic process a linear one. I am continually frustrated with myself and my abilities. But I have gradually gotten better, building on my past experiences and I am not a one trick pony.

Given our fascination with early bloomers, we often forget that some of the best artistic work is due to late bloomers who bring their life experience to the work. There are countless artists whose best work came later in their lives, from Cezanne, to Proust, to Trollope - each of these artists initially seemed failures, but slowly found their voice, through tireless work and experimentation.

I think that the work of late bloomers often seems truer to life because it has this messy process behind it. It reflects life because it has been shaped by the vagaries of life, by the daily joys and the daily tragedies. In contrast, the work of EBs is monolithic and rarely shaped by their circumstances, unless they are financial ones. Often, late bloomers' work feels less artificial than early bloomers' work.

In his essay ``\href{https://www.psychologytoday.com/us/articles/200811/confessions-late-bloomer}{Confessions of a late bloomer},'' Scott Kauffman outlines some essential qualities for late bloomer success. Chiefly \emph{resilience}, \emph{passion} and \emph{perserverance}. In other words, having grit and being tolerant of failure are what's needed for late bloomer success. What late bloomers need is an environment and support to tolerate their brave failures - such support structures are lacking in this day and age. The narrative of a late bloomer's life, unless they are lucky to have rich parents, inevitably includes a tiring and soul sucking day job. Given such odds, is it surprising that so few late bloomers survive?

Ultimately, survival as a late bloomer is a far greater accomplishment than anything an early bloomer has accomplished. It requires a large degree of resilience and internal drive and acceptance of failure.

I am not in favor of one or the other. The world needs both types, but we forget this whenever we think a young prodigy is more marketable than the artist who has years of blood and sweat behind his work. The truth behind what seems effortless is that it requires a lot of revisions to make it seem inevitable and effortless.

To the other late bloomers out there: it is worth enduring for your art. But you have to start somewhere and be willing to throw things away. Things you've toiled long and hard on. Others will probably not understand these pieces. That's okay - hopefully you learned something putting it together. That will have to do. Keep at it.

\hypertarget{resources-6}{%
\section{Resources}\label{resources-6}}

\begin{itemize}
\tightlist
\item
  The book \href{https://davidepstein.com/the-range/}{\emph{Range: How Generalists Triumph in a Specialized World}} is full of a lot of stories of Late Bloomers and how the range of their experiences made them stronger at their work.
\item
  \href{https://www.psychologytoday.com/us/articles/200811/confessions-late-bloomer}{\emph{Confessions of a Late Bloomer}}
\item
  The \href{http://www.newyorker.com/magazine/2008/10/20/late-bloomers-2}{\emph{Malcolm Gladwell essay on Late Bloomers}} is well worth a read as well.
\item
  I started learning Cello at 30, and one of the reasons was the \href{http://cellofun.yuku.com/forums/2/Cellists-by-Night-Forum}{Cellists by Night} discussion board. I don't really read it any more (thanks to some overly pedantic posters), but there are lots of such discussion boards out there for late starters.
\end{itemize}

\hypertarget{on-internal-standards}{%
\chapter{On internal standards}\label{on-internal-standards}}

I have already talked about building your own personal aesthetic, but I haven't yet talked about having standards for when you decide a piece is finished or ready to go out into the world.

The truth is that if you do not define your internal standards by which you judge your work successful, the world will destroy you. Relying on external validation may seem great when you have it, but the tastes of the public changes. What once seemed so appealing about your work may no longer be appealing a year from now. External opinions are ultimately fleeting.

Ira Glass has noted that \href{https://www.brainpickings.org/2014/01/29/ira-glass-success-daniel-sax/}{the gap between execution and your own personal taste} is often the hardest part of starting out as an artist. I find this to be true from bitter experience. We all agonize whether a work is finished or not. Honestly, a work is probably never finished, just abandoned, as goes the old quote. If you are perfectionist, then it will take you a long time to abandon it. But as Ira Glass notes, the more important thing is to just do a lot of work rather than obsess over perfecting each one.

In terms of standards, I believe that perfectionism is largely wasted in a world that only gives art a five-second once over before moving on. The question I always ask myself when I ask whether a work is finished is: \emph{Does the work communicate what I want it to, or is the execution getting in the way?} I'm sure we all know artists who have brilliant ideas, but their execution is lacking. Without proper execution, art feels half-assed. Your friends may say they like it, but maybe they're just being nice. Everything needs to be edited down, pared down, to some extent to expose the strengths of the work.

In terms of execution, there is ultimately a balance that must be struck between precision and passion in every art piece. Clarity and vagueness is another related spectrum. Some of the best pieces of art contain unsolvable mysteries at their center, but these mysteries are always wrapped in substance.

As I've gotten older, I have gotten less anxious about letting my music out into the world. As I've said, opinions are really fleeting, so if I get a bad review I'm much less likely to take it personally. The reviewer rarely has knowledge that I do not about the art. I do my best, and I release music I believe in, and the rest is up to fate.

In the words of Brian Eno: ``My feeling is that a work has little value until you ``release'' it, until you liberate it from yourself and your excuses for it --- ``It's not quite finished yet,'' ''The mix will make all the difference,'' etc. Until you see it out there in the world along with everything else, you don't really know what it is or what to think of it, so it's of no use to you."

\hypertarget{resources-7}{%
\section{Resources}\label{resources-7}}

\href{http://www.salon.com/2011/10/01/david_mitchell_brian_eno/}{This interview of Brian Eno by David Mitchell is really great.}

\hypertarget{on-failure}{%
\chapter{On Failure}\label{on-failure}}

I recently had one of the worst gear failures of my career. My looper, the heart of my performance setup, failed catastrophically just as I was about to start my performance. No matter what I tried, the software betrayed me and didn't work. Luckily, some of my other equipment still worked. And I managed to not only perform, but do some of the best improvisation of my career. Not able to loop my cello playing (which I think of as improvising vertically), I had to improvise horizontally, repeating short sections, alternating with other melodies until it felt I had a song. And people seemed to like it, or at least didn't much care that it wasn't what they expected. The experience overall was tremendously freeing. I realized that I could handle the chaos that life threw at me and make it into art.

After this experience, it occurred to me that if you don't take failure personally, it does interesting things to you. If you accept failure as a circumstance, you may find that you work around failure in interesting ways. In the words of Samuel Beckett: ``Try again. Fail again. Fail better.''

Here are some of my selected thoughts on failure.

\hypertarget{collaborate-with-failure}{%
\section{Collaborate with Failure}\label{collaborate-with-failure}}

As an improviser, I encourage you to collaborate with your failures. Looking at failures in the moment as an external circumstance will enable you to ``use the disadvantage''. Having a background in improvisation allows me to take a \href{https://en.m.wikipedia.org/wiki/\%22Yes,_And\%22_rule}{``Yes, and\ldots{}''} approach to shit as it happens and blows up. I look at it as collaborating with my circumstances.

What I mean is: the audience is forgiving. Take a chance. I long ago decided that a good performance has a chance of failure. I decided that failing in an interesting way was much more important to giving a glossy, perfected performance. It's important to ride the edge of chaos when you perform. Otherwise, you are pretty much a human CD. Over-rehearsing feels way too safe. Riding on the edge of failure makes it much more exciting and engaging for you and the audience.

The most important part of performance is vulnerability. It's also the hardest for us as performers, since we are baring ourselves to possible ridicule. But vulnerability is also how we connect with our audience. Being vulnerable makes us seem human and gives the audience something to connect with.

\begin{quote}
``Many times, the problem for young players is not that they're not great.
It's that they're afraid to go to these places, to take themselves
emotionally to places that may be embarrassing, that may involve
some sort of personal failure. It's one thing to say, ``It's happy music.
Then I'm going to use more vibrato or something.'' No.~The first step
is to be able to take yourself to that place. And the hardest thing is
that you have to do that over and over and over."
- Geoff Nuttall, St.~Lawrence String Quartet
\end{quote}

\hypertarget{make-your-failures-interesting---go-for-fiasco}{%
\section{Make your failures interesting - go for fiasco}\label{make-your-failures-interesting---go-for-fiasco}}

\begin{quote}
Mistakes are not just opportunities for learning; they are, in an important sense,
the only opportunity for learning or making something truly new. - Daniel Dennett
\end{quote}

\begin{quote}
Don't search for the answers, which could not be given to you now, because you
would not be able to live them. And the point is to live everything. Live the questions now. - Rilke
\end{quote}

I think it's important is that you should be pushing yourself when you make art. You need to push yourself in order to avoid repeating yourself. You should be striving for an artistic fiasco. Taking such risks makes art into a high-risk/high-reward venture. The critics may not take kindly to your (temporary) insanity, but that's just one point of view. Touching and moving people with your art requires taking risks and that is worth braving a few failures for. However, there has to be a reason for your risk beyond mere provocation. Provoking the audience only works one or two times in an artist's career before the audience realizes the artist is crying wolf.

It is also important to not become complacent with your work. Strive to learn new things and gain new experiences everyday. These experiences don't have to be big ones, but they should push your boundaries. Be bad at something. Let your inexperience and failure in these new experiences humble and recenter you. Realize that failure is not the end of the world.

\hypertarget{dont-take-failure-personally---avoid-rumination}{%
\section{Don't take failure personally - avoid rumination}\label{dont-take-failure-personally---avoid-rumination}}

I have to emphasize that you shouldn't be taking failure personally. Many circumstances contributing to a failure are beyond your control: perhaps your work doesn't capture the gestalt, or it was released at the wrong time. Actors face rejection all the time when going for auditions, and it is bruising. As artists and performers, we often subject ourselves to the crushing weight of expectations. Paraphrasing the piano teacher Seymour Bernstein, many great artists/performers are assholes because they can achieve perfection in the world of performance but then have to deal with the fact that the world is not so perfect. Their expectations don't align with the true reality of the world.

Managing failure is really about managing your expectations. I believe that much of what we term as writer's block is such a fear of failure and unrealistic expectations of our work. To quote Sarah Lewis, ``\href{https://www.goodreads.com/book/show/18143786-the-rise}{Failure is an orphan until we give it a narrative.}'' I understand that when we are young, we believe that our art can change the world. However, I think that the best we can strive for as we grow as artists is to cultivate gratitude for those who are receptive to our art, and to not feel entitled to a ``wide audience''. Honestly, I feel such an audience does not really exist any more.

It is also important not to endlessly obsess about your failure and what you could have done differently. Ruminating about failure leads to depression, which can lead to abandoning your art. Instead of admonishing yourself and beating yourself up, try reframing it as a new challenge. Instead of ``That was awful, no one liked it. I'm not a good artist'', you should be telling yourself, ``well, that didn't work. What if I tried this instead?''

If done right, an artistic career is full of failures. The successes are few and far between, but are that much more sweeter for that reason. But these failures are opportunities to learn, not to beat yourself up. Beating yourself up will only make you risk-averse, and make your art too safe.

\hypertarget{endure-failure-and-grow-from-it}{%
\section{Endure failure and grow from it}\label{endure-failure-and-grow-from-it}}

So your work was panned. Critics did not get the meaning of your work. Or even worse, your work, which you put years of work and your heart and soul into, was received by the world with a resounding meh. For most artists who don't live the charmed life of being in the spotlight, failure is a more common state than success. Or is it really failure? I believe that getting a work to the point at which it is released is a monumental accomplishment. Acknowledge the fact that you put effort into the work. Feel secure that you did everything you could under to do your best work under the circumstances.

Get this into your head. There is nothing wrong with you because your work did not make a big splash. There was usually a cultural mismatch - it came out at the wrong time, or there was a work with more hype that overshadowed it. The initial response to a work of art is misleading. Your work may yet find an audience over a period of time.

\hypertarget{our-connections-will-sustain-us}{%
\section{Our connections will sustain us}\label{our-connections-will-sustain-us}}

I have written before about the importance of building an artistic community. Part of failing well is to fail safely by building a community where you are supported. When you fail, it is these connections that will sustain you through your failures to try again. If you have been supportive of others, now is the time that they will give back to you. But sometimes you will have to ask for that support.

I am a big believer in the community created by art. The value of an artistic piece in my opinion is how much it inspires others to do art, not the monetary value.

\hypertarget{resources-8}{%
\section{Resources}\label{resources-8}}

\begin{itemize}
\tightlist
\item
  Sarah Lewis' \href{https://www.goodreads.com/book/show/18143786-the-rise}{The Rise: Creativity, the Gift of Failure, and the Search for Mastery} is worth reading.
\item
  The NEA's issue on \href{https://www.arts.gov/sites/default/files/nea_arts/NEA\%20Arts_2014_no4_web.pdf}{Art and Failure} is a wonderful compilation of artists of all types talking about the role failure has played in their careers.
\end{itemize}

\hypertarget{on-self-doubt}{%
\chapter{On Self Doubt}\label{on-self-doubt}}

I think that every artist has periods of self-doubt. Despite releasing multiple albums, despite having composed music for nearly twenty years, there are still the demons of self doubt in my head. I can still often be \emph{the boy who takes everything personally}.

My most recent round of self doubt started with dissatisfaction with my cello tone. Although ultimately I'm an amateur cellist, it's hard not to compare my sound with more professional musicians. Why is it not as rich and varied? It must be something to do with me. I'm just not good enough, I tell myself.

I have written before about how it's \href{on-success-or-how-to-suck-it-up-and-be-happy-for-your-successful-friends.html}{counterproductive to compare yourself to others}, but I can't help it sometimes. When I'm in this state, listening to other people's music is painful to me - each note can feel like a rebuke, and I can't be happy for my friends' accomplishments. It's a bad place to be.

The worst of it is when external opinions sync with one's own self-critical voice. If there's a review that \href{https://www.fluid-radio.co.uk/2017/07/marcus-fischer-film-variations/}{calls my playing ``too swooningly romantic''}, I take it as validation of my own negative voice. I am so used to minimizing myself, wanting to disappear when that happens. I sometimes feel like I'm an albatross around the neck of my collaborators. So I disappear, stop connecting with people, stop going to shows. And I'm left struggling in the morass of self doubt.

What is the way back? For me, I think it has to do with three things: 1) reconnect with the people who value my work, 2) be more self compassionate, and 3) remember that I do art not for the external validation, but because the journey and process is in itself rewarding. I need to re-embrace my strengths as an amateur cellist. I can be adventurous in ways that other cellists can't. I need to embrace the ugliness and simplicity of my performances. My cello playing can sound different, since I don't play so that I can pay the rent and eat. I can experiment.

The self compassion is the hardest part, especially after a childhood built on extreme self criticism. It can be hard not to replay these old tapes. I need to find new, more positive ways of motivation. At the same time, I have to be less precious of my art. My albums are documents of what I was feeling and what I was obsessed with at the time.

Being an artist means taking part in traditions and customs established by one's predecessors and peers. It's important to take part in this creative dialogue, but you have to be careful that it doesn't become a monologue, either from the tradition side or your side. To err on the side of tradition means minimizing one's art; to err on out own side means missing opportunities to make connections with others.

I think that we forget that it takes a certain bravery to be an artist, especially in such an age of anonymous reviews and rapid, unconsidered responses. It is a brave act to take something that is personal to us, part of our own internal mythology, and show it to the world with the hope that it connects with someone. How easy it is to feel shut down if there's one naysayer - or even worse, if there is no response. In such times, we must look into ourselves and ask ourself the essential question: \emph{why do I make art?} We must disabuse ourselves of any romantic notions and truly embrace the struggle and joy of making art.

\end{document}
